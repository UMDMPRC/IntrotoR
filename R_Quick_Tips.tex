\documentclass{beamer}
\usepackage[utf8]{inputenc}
\usepackage{verbatim}
\usepackage{tikz}
\usepackage{hyperref}
\setbeamertemplate{footline}[frame number]
\title{Quick R Tips}
\author{\texorpdfstring{Heide Jackson \newline\url{heidej@umd.edu}}{Author}}
\institute{University of Maryland Population Research Center}

\date{August 2019}

\begin{document}
\maketitle
\begin{frame}{High Level Things to Know}
\begin{itemize}
\item R is free and always will be.
\item It's a flexible statistical software environment based on the S language.
\item R is dynamic and updated more frequently than other statistical software packages like SAS or Stata.
\item R is user driven and its functionality is greatly enhanced by user contributed packages.
\item R Studio is a nice wrapper for accessing R
\end{itemize}
\end{frame}


\begin{frame}{Where to Find R and R Studio}

\item R can be downloaded at: \url{https://cran.r-project.org/}.

\item R Studio can be downloaded at: \url{https://rstudio.com/products/rstudio/download/}.

\end{frame}

\begin{frame}[fragile]{The basics of doing anything in R}
\begin{verbatim}
object<-function(...) #Frequent R users like this method

or equivalently

object=function()
\end{verbatim}
\begin{itemize}
    \item Functions may be built into R's basic system, loaded in from packages, or defined by you, the user.
\end{itemize}
\end{frame}

\begin{frame}{Other Things To Know}
\begin{itemize}
    \item R is case sensitive. E.g. x and X will be stored in R as two different objects.
    \item The help function is your friend.  If you are trying to figure out what a function does, type help(functioname) or equivalently ?functionname.
    \item Objects can be embedded in other objects.  If you call data\$x, this can be different from calling x.
   \item \# is used prior to a comment.  Multi line comments aren't supported but there are system and program specific shorts cuts for this if you want them.
    \end{itemize}
    \end{frame}


\begin{frame}{Common Hazards and Missteps of Working in R}{Redundancy}
\begin{itemize}

    \item Redundancy is major limitation and benefit to working in R.
    \begin{itemize}
        \item Any task, no matter how simple or mundane can be accomplished at least five different ways using R or user supplied packages.
        \item Users may be overwhelmed by choice. 
        \item The benefit to this is any task can be effectively dual coded.  When in doubt about what a function is doing, look for another package that does the same thing.
        \item When trying to decide how to code, it's sometimes worth doing a task in the easiest way and (if not too difficult) a method that uses just base R.  
    \end{itemize}
\end{itemize}
\end{frame}

\begin{frame}{Common Hazards and Missteps of Working in R}{Quality Control}
\begin{itemize}
    \item Great R programmers may not be great researchers and vice versa.
    \item How do you evaluate the quality of an R Package?
    \begin{itemize}
        \item For common tasks, packages that have been more extensively used tend to be better debugged.  
        \item Right now the tidyverse collection of packages has  become the R version of mainstream.
        \item For complex methods, research the contributor.  Where have they published?  What documentation do they provide?  What vignettes do they create?  What data do they use?
        \item When in doubt, check your work.  Can estimation in R be reproduced in other statistical programs (such as Stata or SAS).  
    \end{itemize}
\end{itemize}
\end{frame}

\begin{frame}{Common Hazards and Missteps of Working in R}{Version Control}
\begin{itemize}
    \item R and its packages are frequently updated.  R code may work one day and crash the next.
    \item Package control packages can help resolve this.  A couple of options to consider are docker and packrat.
    \item As a global disclaimer, just because code is reproducible doesn't mean that it is correct.  
    
\end{itemize}
\end{frame}

\end{document}
