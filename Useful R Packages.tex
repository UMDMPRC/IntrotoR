\documentclass{report}
\usepackage[utf8]{inputenc}
\setcounter{secnumdepth}{0}
\usepackage{hyperref}
\hypersetup{
    colorlinks=true,
    linkcolor=blue,
    filecolor=magenta,      
    urlcolor=cyan,
}
\title{Useful R Packages}
\author{Heide Jackson 
\\University of Maryland Population Research Center\\ \url{heidej@umd.edu}}

\date{September 2019}

\begin{document}

\maketitle
\tableofcontents
\pagebreak
\section*{Disclaimers}
R and R packages are user developed and user maintained.  Some R packages may not be compatible with others and some R functions may not work on all versions of R.  

\section{Installing and Using Packages}
Using packages within an R session requires two steps:
\begin{enumerate}
    \item Installation 
    \item Loading
\end{enumerate}
Below is a quick example of R code to install and load a package.
\begin{verbatim}
install.packages("Hmisc") #needed once
library("Hmisc") # needed every time you want to use the package in a session
\end{verbatim}
Typically, installation will only need to be done once, but the library function will be needed every time you want to use a package's functions within your R session.
\section{Version Control}
R and R packages are dynamic. Sometimes if you want code to remain reproducible, it may be of interest to use an R package that maintains version control. 

\pagebreak
\section{Useful R Packages}
\subsection{General Use Packages}
\begin{itemize}

    \item \href{https://www.tidyverse.org/}{Tidyverse} a collection of packages for cleaning, manipulating, and visualizing data.  The grammar of functions in the tidyverse is very similar to Pandas in python.   
    
    As mentioned, Tidyverse is a collection of packages.  Installing Tidyverse will download all of these packages, but packages are best loaded individually.  Below is a list of packages in the tidyverse I find particularly helpful:
    \begin{itemize}
        \item \href{https://haven.tidyverse.org/}{haven} Reads in and writes data from other statistical packages. My experience is this works better for Stata than SAS.
        \item \href{https://readr.tidyverse.org/}{readr} Reads in and writes rectangular data sets.  Typically, these are csv or text files.
        \item \href{https://cran.r-project.org/web/packages/dplyr/vignettes/dplyr.html}{dplyr} Manipulates data in various ways.  Contains functions to keep, drop, reorganize, and modify data.
        \item 
        \href{https://ggplot2.tidyverse.org/}{ggplot2}  THE package for data visualization in R.
    \end{itemize}
    
    \item \href{https://cran.r-project.org/web/packages/stargazer/vignettes/stargazer.pdf}{stargazer} Provides the ability to generate pretty tables from R output.
    \item \href{https://cran.r-project.org/web/packages/Hmisc/Hmisc.pdf}{Hmisc} An assortment of miscellaneous packages that assist in data manipulation and analysis.
    
    \end{itemize}
    \subsection{Version Control Packages}
    \begin{itemize}
    \item \href{https://rviews.rstudio.com/2018/01/18/package-management-for-reproducible-r-code/}{Useful information} about R version control and different options on how to go about it.
    \item \href{https://cran.r-project.org/web/packages/packrat/packrat.pdf}{packrat} Version control package that I personally like.
    \item \href{https://docs.docker.com/install/}{docker} Another version control option that comes recommended.
    \end{itemize}
    \subsection{Multiple Imputation Packages}
    \begin{itemize}
       \item  \href{https://www.rdocumentation.org/packages/mice/versions/3.6.0/topics/mice|}{mice} Offers different options for imputing data in R.
       \item \href{https://cran.r-project.org/web/packages/mitools/mitools.pdf}{mitools} Offers tools for working with multiply imputed data in R.
       \end{itemize}
       \subsection{Survey Packages}
       \begin{itemize}
      \item \href{https://cran.r-project.org/web/packages/survey/survey.pdf}{survey} Provides functions for working with survey data and incorporating survey weights into analysis.
      \end{itemize}
      \subsection{Statistical Methods Packages}
      \begin{itemize}
      \item \href{https://cran.r-project.org/web/packages/nnet/nnet.pdf}{nnet} Functions for running multinomial logistic regression in R. 
      \item \href{https://cran.r-project.org/web/packages/survival/survival.pdf}{survival} Provides functions for fitting and evaluating single state survival models (event history analysis) in R.
      \item \href{https://cran.r-project.org/web/packages/multistate/multistate.pdf}{multistate} Provides functions for fitting and evaluating multistate models in R.
      \item \href{https://cran.r-project.org/web/packages/boot/boot.pdf}{boot} Bootstraps data in R.
      \end{itemize}
\subsection{Data Visualization Packages}
\begin{itemize}
\item   \href{https://ggplot2.tidyverse.org/}{ggplot2}  THE package for data visualization in R.  A lot of people like and use this package.  I like it mainly for difficult or complex graphics.
\item \href{https://cran.r-project.org/web/packages/lattice/lattice.pdf}{lattice} Offers function for making multi-panel graphs.
\item \href{https://cran.r-project.org/web/packages/RColorBrewer/RColorBrewer.pdf}{RColorBrewer} Allows for custom color creation in R.
    

\end{itemize}
\end{document}
